% Created 2016-06-29 Wed 13:09
\documentclass[11pt]{article}
\usepackage[utf8]{inputenc}
\usepackage[T1]{fontenc}
\usepackage{fixltx2e}
\usepackage{graphicx}
\usepackage{grffile}
\usepackage{longtable}
\usepackage{wrapfig}
\usepackage{rotating}
\usepackage[normalem]{ulem}
\usepackage{amsmath}
\usepackage{textcomp}
\usepackage{amssymb}
\usepackage{capt-of}
\usepackage{hyperref}
\author{Picaud}
\date{\today}
\title{Implementing the Levenberg-Marquardt method}
\hypersetup{
 pdfauthor={Picaud},
 pdftitle={Implementing the Levenberg-Marquardt method},
 pdfkeywords={},
 pdfsubject={},
 pdfcreator={Emacs 24.5.1 (Org mode 8.3.4)}, 
 pdflang={English}}
\begin{document}

\maketitle

\section*{Problem description}
\label{sec:orgheadline1}

This is a \textbf{nonlinear least squares} problem

\[
\theta = \arg \min \limits_\theta \frac{1}{2}\|f(\theta)\|^2_2
\]

\section*{Mathetamica test}
\label{sec:orgheadline6}

\subsection*{Preamble}
\label{sec:orgheadline5}

\subsubsection*{Gradient computation can be done as follow}
\label{sec:orgheadline2}

\begin{verbatim}
grad[f_, var_List] := D[f, #] & /@ var;
\end{verbatim}

\subsubsection*{Function to fit}
\label{sec:orgheadline3}

\begin{verbatim}
(* Function to fit  *)
f[x_]:=a0+a1*x+a2*x*x;

(* Parameters  *)
theta={a0,a1,a2};

(* Parameter value *)
thetaTrueValue={1,2,-1};
\end{verbatim}

\subsubsection*{Data}
\label{sec:orgheadline4}
\begin{verbatim}
nSample = 10;
xSample = Table[i/nSample,{i,0,nSample-1}];
ySample = Table[f[xSample[[i]]]  /. Thread[theta ->thetaTrueValue],{i,1,nSample}];
\end{verbatim}


\begin{verbatim}
grad[f_, var_List] := D[f, #] & /@ var;
(* Function to fit  *)
f[x_]:=a0+a1*x+a2*x*x;

(* Parameters  *)
theta={a0,a1,a2};

(* Parameter value *)
thetaTrueValue={1,2,-1};
nSample = 10;
xSample = Table[i/nSample,{i,0,nSample-1}];
ySample = Table[f[xSample[[i]]]  /. Thread[theta ->thetaTrueValue],{i,1,nSample}];

\[Theta] = ySample
\end{verbatim}

\begin{center}
\begin{tabular}{rlllllllll}
1 & 119/100 & 34/25 & 151/100 & 41/25 & 7/4 & 46/25 & 191/100 & 49/25 & 199/100\\
\end{tabular}
\end{center}


\begin{verbatim}
(* Gradient Computation *)
grad[f_, var_List] := D[f, #] & /@ var;

(* Function to fit *)
f[x_] := a0 + a1*x + a2*x*x;

(* Parameters *)
theta = {a0, a1, a2};

(* Generate data to fit from theta true value *)

thetaTrueValue = {1, 2, -1};
nSample = 10;
xSample = Table[i/nSample, {i, 0, nSample - 1}];
ySample = 
  Table[f[xSample[[i]]] /. Thread[theta -> thetaTrueValue], {i, 1, 
    nSample}];

(* Define objective function *)

evalObjectiveGradHessian[f_] := 
  Module[{F, dF}, 
   F = Table[ySample[[i]] - f[xSample[[i]]], {i, 1, nSample}]; 
   dF = grad[F, theta]; Return[{F.F/2, dF.F, dF.Transpose[dF]}]];

evalObjectiveGradHessian[f]
\end{verbatim}

\begin{center}
\begin{tabular}{llll}
((1 - a0) \^{}2 + (199/100 - a0 - (9*a1) /10 - (81*a2) /100) \^{}2 + (49/25 - a0 - (4*a1) /5 - (16*a2) /25) \^{}2 + (191/100 - a0 - (7*a1) /10 - (49*a2) /100) \^{}2 + (46/25 - a0 - (3*a1) /5 - (9*a2) /25) \^{}2 + (7/4 - a0 - a1/2 - a2/4) \^{}2 + (41/25 - a0 - (2*a1) /5 - (4*a2) /25) \^{}2 + (151/100 - a0 - (3*a1) /10 - (9*a2) /100) \^{}2 + (34/25 - a0 - a1/5 - a2/25) \^{}2 + (119/100 - a0 - a1/10 - a2/100) \^{}2) & /2 & (-323/20 + 10*a0 + (9*a1) /2 + (57*a2) /20 (-9* (199/100 - a0 - (9*a1) /10 - (81*a2) /100)) /10 - (4* (49/25 - a0 - (4*a1) /5 - (16*a2) /25)) /5 - (7* (191/100 - a0 - (7*a1) /10 - (49*a2) /100)) /10 - (3* (46/25 - a0 - (3*a1) /5 - (9*a2) /25)) /5 - (2* (41/25 - a0 - (2*a1) /5 - (4*a2) /25)) /5 - (3* (151/100 - a0 - (3*a1) /10 - (9*a2) /100)) /10 + (-119/100 + a0 + a1/10 + a2/100) /10 + (-34/25 + a0 + a1/5 + a2/25) /5 + (-7/4 + a0 + a1/2 + a2/4) /2 (-81* (199/100 - a0 - (9*a1) /10 - (81*a2) /100)) /100 - (16* (49/25 - a0 - (4*a1) /5 - (16*a2) /25)) /25 - (49* (191/100 - a0 - (7*a1) /10 - (49*a2) /100)) /100 - (9* (46/25 - a0 - (3*a1) /5 - (9*a2) /25)) /25 - (4* (41/25 - a0 - (2*a1) /5 - (4*a2) /25)) /25 - (9* (151/100 - a0 - (3*a1) /10 - (9*a2) /100)) /100 + (-119/100 + a0 + a1/10 + a2/100) /100 + (-34/25 + a0 + a1/5 + a2/25) /25 + (-7/4 + a0 + a1/2 + a2/4) /4) & ((10 9/2 57/20) (9/2 57/20 81/40) (57/20 81/40 15333/10000))\\
\end{tabular}
\end{center}
\end{document}